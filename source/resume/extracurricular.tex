\cvsection{Extracurricular}
\begin{cventries}
  \cventry
    {\href{https://github.com/acdemiralp/gl}{https://github.com/acdemiralp/gl}}
    {GL}
    {C++}
    {2017}
    {
      \begin{cvitems}
        \item {An OpenGL 4.6 Core Profile wrapper written in C++11.}
        \item {Featured on the Khronos website:  \href{https://www.khronos.org/news/archives/2017/08/}{https://www.khronos.org/news/archives/2017/08/}. }
      \end{cvitems}
    }
    
  \cventry
    {\href{https://github.com/acdemiralp/nano_engine}{https://github.com/acdemiralp/nano\_engine}}
    {Nano Engine}
    {C++}
    {2017}
    {
      \begin{cvitems}
        \item {A concise platform-independent display, input and graphics abstraction based on SDL2.}
      \end{cvitems}
    }
     
  \cventry
    {\href{https://github.com/acdemiralp/vulkan_sdl}{https://github.com/acdemiralp/vulkan\_sdl}}
    {Vulkan SDL}
    {C++}
    {2017}
    {
      \begin{cvitems}
        \item {An utility for creating Vulkan surfaces platform-independently based on SDL2.}
        \item {Supports all windowing systems supported by Vulkan: Android, Mir, Wayland, Win32, XCB, Xlib.}
      \end{cvitems}
    }
   
  \cventry
    {\href{https://github.com/acdemiralp/unity_data_binding}{https://github.com/acdemiralp/unity\_data\_binding}}
    {Unity Data Binding}
    {C\#}
    {2015}
    {
      \begin{cvitems}
        \item {A data and event binding plugin for Unity 4.6 and above. The implementation is loosely based on .NET System.Windows.Data.}
        \item {Bundled with Unity editor scripts, making it possible to use without writing additional code.}
      \end{cvitems}
    } 
  \cventry
    {\href{https://github.com/acdemiralp/framegraph}{https://github.com/acdemiralp/framegraph (Private until release)}}
    {Framegraph}
    {C++}
    {In Progress}
    {
      \begin{cvitems}
        \item {A high-level rendering abstraction which describes a frame as a directed acyclic graph consisting of render passes and resources.}
        \item {Based on the Game Developers Conference (GDC) presentation by Yuriy O'Donnell on EA Frostbite's rendering architecture.}
      \end{cvitems}
    }
   
  \cventry
    {\href{https://github.com/acdemiralp/makina}{https://github.com/acdemiralp/makina (Private until release)}}
    {Makina}
    {C++}
    {In Progress}
    {
      \begin{cvitems}
        \item {A "not-game engine" which provides a subset of common game engine features such as audio, input, physics, rendering, scripting but in contrary to most game engines, is easily extendable with prototypical computer graphics and virtual reality research ideas.}
      \end{cvitems}
    }
    
\end{cventries}
